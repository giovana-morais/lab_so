\documentclass[12pt]{article}
\usepackage[utf8]{inputenc}
\usepackage[brazilian]{babel}
\usepackage{enumitem}

\author{Gabriel de Paula e Lima  587710\\
        Giovana Vieira de Morais  587591}
\title{Relatório Projeto 3}
\begin{document}

\maketitle
\newpage

\section*{A atividade}

\begin{description}[labelindent=1cm]
    \item[Compilar o módulo fornecido como exemplo]
    \item[Modificar o módulo fornecido para exibir, no lugar da frase fixa, o
        PID do processo lendo o arquivo e o PID do seu processo pai]
    \item[Dar ao interpretador de comando executando o processo de leitura
        permissões de root]
\end{description}

\subsection*{Exibir o PID do processo lendo o arquivo (\texttt{cat}) e do
processo pai}
    Para essa função, foi necessário usar a task\_struct, que é uma struct que
    descreve informações de processos ou tarefas do sistema, guardando
    informações importantes como PID, nome do processo atual, credenciais do
    grupo e do processo e processo pai.

    Após a declaração de um ponteiro para a estrutura, só foi necessário
    imprimir o nome (\texttt{task->comm}) e o pid (\texttt{task->pid}) do
    processo atual. Para imprimir as informações do processo pai:
    \texttt{task->parent->comm} e \texttt{task->parent->pid}.
\subsection*{Dar ao processo pai permissões de root}
    Assim como mencionado acima, a \texttt{task\_struct} contém em suas
    informações as credenciais do usuário e do grupo. 
\end{document}
